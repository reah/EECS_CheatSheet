\documentclass[10pt,landscape]{article}
\usepackage{multicol}
\usepackage{calc}
\usepackage{ifthen}
\usepackage[landscape]{geometry}
\usepackage{amsmath,amsthm,amsfonts,amssymb}
\usepackage{color,graphicx,overpic}
\usepackage{hyperref}


\pdfinfo{
  /Title (example.pdf)
  /Creator (TeX)
  /Producer (pdfTeX 1.40.0)
  /Author (Seamus)
  /Subject (Example)
  /Keywords (pdflatex, latex,pdftex,tex)}

% This sets page margins to .5 inch if using letter paper, and to 1cm
% if using A4 paper. (This probably isn't strictly necessary.)
% If using another size paper, use default 1cm margins.
\ifthenelse{\lengthtest { \paperwidth = 11in}}
    { \geometry{top=.5in,left=.5in,right=.5in,bottom=.5in} }
    {\ifthenelse{ \lengthtest{ \paperwidth = 297mm}}
        {\geometry{top=1cm,left=1cm,right=1cm,bottom=1cm} }
        {\geometry{top=1cm,left=1cm,right=1cm,bottom=1cm} }
    }

% Turn off header and footer
\pagestyle{empty}

% Redefine section commands to use less space
\makeatletter
\renewcommand{\section}{\@startsection{section}{1}{0mm}%
                                {-1ex plus -.5ex minus -.2ex}%
                                {0.5ex plus .2ex}%x
                                {\normalfont\large\bfseries}}
\renewcommand{\subsection}{\@startsection{subsection}{2}{0mm}%
                                {-1explus -.5ex minus -.2ex}%
                                {0.5ex plus .2ex}%
                                {\normalfont\normalsize\bfseries}}
\renewcommand{\subsubsection}{\@startsection{subsubsection}{3}{0mm}%
                                {-1ex plus -.5ex minus -.2ex}%
                                {1ex plus .2ex}%
                                {\normalfont\small\bfseries}}
\makeatother

% Define BibTeX command
\def\BibTeX{{\rm B\kern-.05em{\sc i\kern-.025em b}\kern-.08em
    T\kern-.1667em\lower.7ex\hbox{E}\kern-.125emX}}

% Don't print section numbers
\setcounter{secnumdepth}{0}


\setlength{\parindent}{0pt}
\setlength{\parskip}{0pt plus 0.5ex}

%My Environments
\newtheorem{example}[section]{Example}
% -----------------------------------------------------------------------

\begin{document}
\raggedright
\footnotesize
\begin{multicols}{3}


% multicol parameters
% These lengths are set only within the two main columns
%\setlength{\columnseprule}{0.25pt}
\setlength{\premulticols}{1pt}
\setlength{\postmulticols}{1pt}
\setlength{\multicolsep}{1pt}
\setlength{\columnsep}{2pt}

{\scriptsize\subsection{ERG C100}
\hspace{5pt} $\cdot$ {\bf Capacity Factor} - the ratio of the actual output of a power plant over a given period of time to its potential output if it were to operate at maximum capacity over that same time interval. Capacity factor describes to what degree a power plant operates (how often or at what output level).\\{\bf Total Energy = PowerOutput(Nameplate Capacity) $\times$ Time $\times$ CapacityFactor}

\hspace{5pt} $\cdot$ {\bf 1$^{st}$ Law Efficiency} - the ratio of useful work produced (electricity, or electricity and purposeful heating in the case of cogeneration) to the amount of primary energy put in to the power plant to produce that useful work. Efficiency describes how well a power plant converts energy from one form into another.\\}

\hspace{5pt} $\cdot$ {\bf High Voltage Transmission} - high voltage is used for transmission of electricity because P$_{loss} = I^{2}\times$R, which means that line losses increase at the current over the line goes up. Thus, we�d like to reduce current to reduce line loses. Power = Voltage x Current, meaning that if power flow is fixed, there is an inverse relationship between voltage and current. If we want a smaller current value at a fixed power flow, then the voltage has to increase. In sum, high voltage means low current, which in turn means smaller transmission loses.

\hspace{5pt} $\cdot$ {\bf Benefits/Pitfalls of using Biofuels for Transportation} - {\it potential benefits:} Lower lifecycle carbon emissions (cellulosic ethanol), reduced dependence on foreign oil (US context), liquid fuels can work with existing vehicles and fueling infrastructure.\\
{\it potential pitfalls:} Competition between food and fuel (increased food prices), indirect land use change, depletion of soil nutrients or increased use of artificial fertilizers

\hspace{5pt} $\cdot$ {\bf Electricity industry's development from Edison�s small decentralized utility model to one of large centralized power stations} - Scales of economy and improving efficiencies meant that electricity could be produced more cheaply by larger turbines.The development of alternating current allowed electricity to be transmitted over longer distances with less loss.

% You can even have references
\rule{0.3\linewidth}{0.25pt}
\scriptsize
\bibliographystyle{abstract}
\bibliography{refFile}
\end{multicols}
\end{document}