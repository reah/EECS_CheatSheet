\documentclass[10pt,landscape]{article}
\usepackage{multicol}
\usepackage{calc}
\usepackage{ifthen}
\usepackage[landscape]{geometry}
\usepackage{amsmath,amsthm,amsfonts,amssymb,courier}
\usepackage{color,graphicx,overpic}
\usepackage{hyperref}


\pdfinfo{
  /Title (example.pdf)
  /Creator (TeX)
  /Producer (pdfTeX 1.40.0)
  /Author (Seamus)
  /Subject (Example)
  /Keywords (pdflatex, latex,pdftex,tex)}

% This sets page margins to .5 inch if using letter paper, and to 1cm
% if using A4 paper. (This probably isn't strictly necessary.)
% If using another size paper, use default 1cm margins.
\ifthenelse{\lengthtest { \paperwidth = 11in}}
    { \geometry{top=.25in,left=.25in,right=.25in,bottom=.25in} }
    {\ifthenelse{ \lengthtest{ \paperwidth = 297mm}}
        {\geometry{top=1cm,left=1cm,right=1cm,bottom=1cm} }
        {\geometry{top=1cm,left=1cm,right=1cm,bottom=1cm} }
    }

% Turn off header and footer
\pagestyle{empty}

% Redefine section commands to use less space
\makeatletter
\renewcommand{\section}{\@startsection{section}{1}{0mm}%
                                {-1ex plus -.5ex minus -.2ex}%
                                {0.5ex plus .2ex}%x
                                {\normalfont\large\bfseries}}
\renewcommand{\subsection}{\@startsection{subsection}{2}{0mm}%
                                {-1explus -.5ex minus -.2ex}%
                                {0.5ex plus .2ex}%
                                {\normalfont\normalsize\bfseries}}
\renewcommand{\subsubsection}{\@startsection{subsubsection}{3}{0mm}%
                                {-1ex plus -.5ex minus -.2ex}%
                                {1ex plus .2ex}%
                                {\normalfont\small\bfseries}}
\makeatother

% Define BibTeX command
\def\BibTeX{{\rm B\kern-.05em{\sc i\kern-.025em b}\kern-.08em
    T\kern-.1667em\lower.7ex\hbox{E}\kern-.125emX}}

% Don't print section numbers
\setcounter{secnumdepth}{0}


\setlength{\parindent}{0pt}
\setlength{\parskip}{0pt plus 0.5ex}

%My Environments
\newtheorem{example}[section]{Example}
% -----------------------------------------------------------------------

\begin{document}
\raggedright
\footnotesize
\begin{multicols}{3}


% multicol parameters
% These lengths are set only within the two main columns
%\setlength{\columnseprule}{0.25pt}
\setlength{\premulticols}{1pt}
\setlength{\postmulticols}{1pt}
\setlength{\multicolsep}{1pt}
\setlength{\columnsep}{2pt}

%\begin{center}
%     \Large{\underline{CS 61C}} \\
%\end{center}

\subsection{Modular Arithmetic}
\hspace{5pt} $\cdot$ Addition: O(n)\\
\hspace{5pt} $\cdot$ Multiplication: O(n$^{2}$) {\it (naive)}\\
\hspace{5pt} $\cdot$ Multiplication: O(nlogn) {\it (FFT)}\\
\hspace{5pt} $\cdot$ Euclid's Rule: gcd(x, y) = gcd(x mod y, y)\\
\hspace{5pt} $\cdot$ \# of bits in x$^{y}$ = ylog$_{2}$x $\leq$ n$\cdot$2$^{n}$ \\
\vspace{1pt}\hspace{5pt} $\cdot$ $\frac{n}{2}^{\frac{n}{2}} \leq n! \leq n^{n}$\\
\vspace{1pt}\hspace{5pt} $\cdot$ {\it f:} S $\rightarrow$ T {\scriptsize is 1-to-1 (injective) \& onto (surjective) $\Rightarrow$ $\mid S \mid = \mid T \mid$}\\
\hspace{5pt} $\cdot$ {\it f:} S $\rightarrow$ T is 1-to-1 (injective) $\Rightarrow$ $\mid T \mid \geq \mid S \mid$\\
\hspace{5pt} $\cdot$ $\sum_{i=0}^{\infty}$ r$^{i}$ = $\frac{1}{1-r}$, if r $\textless$ 1\\
\hspace{5pt} $\cdot$ $\sum_{i=0}^{n}$ i$^{2}$ = $\frac{n(n+1)(2n+1)}{6}$\\
\hspace{5pt} $\cdot$ $\sum_{i=0}^{n}$ $\frac{1}{i}$ = O(log$_{2}$n)\\

\subsubsection{Extended Euclid's GCD(x,y)}
O(n$^{3}$); gcd(x,y) = d = x{\it i} + y{\it b}; x $\geq$ y; \# mod x\\
 \texttt{\textbf{ext-gcd}(x,y):\\ 
 \hspace{3pt} if y == 0: return (x, 1, 0)\\ 
 \hspace{3pt} else:\\ 
 \hspace{10pt} (d, a, b) = ext-gcd(y, x mod y)\\
 \hspace{10pt} return (d, b, a-$\frac{x}{y}\cdot b$)}
 \includegraphics[scale=.45]{gcd.png}
 
 \subsubsection{Fermat's Little Theorem}
 if p is prime, then $\forall$ 1 $\leq$ a $\textless$ p\\
\hspace{5pt} a$^{p-1}$ = 1 mod p \\
\textit{\textbf{Proof:}}{\it Start by listing first p-1 positive multiples of a:\\
S = \{a, 2a, 3a, $\cdots$ (p -1)a\}\\
Suppose that {\it r}a and {\it s}a are the same mod p, $\Rightarrow$ {\it r = s} mod p\\
$\therefore$ set S of p-1 multiples of a are distinct and nonzero, that is, they must be congruent to 1, 2, 3, $\cdots$ p-1 after being sorted.\\
Multiply all congruences together and we find
a$\cdot$2a$\cdot$3a$\cdots$(p-1)$\cdot$a = 1$\cdot$2$\cdot$3$\cdots$(p-1) (mod p)
or better, a$^{(p-1)}$(p-1)! = (p-1)! mod p. Divide both side by (p-1)! $\blacksquare$}\\

\subsubsection{Primality Testing}
\begin{flushleft}\vspace{-16pt}\[
\hspace{-50pt}any \ a  \rightarrow a^{N-1} = 1 \ mod \ N {\it ?}
\begin{cases}
	{\it yes} \Rightarrow {\it "prime"}\\
	{\it no} \Rightarrow {\it composite}
\end{cases}
\]
\end{flushleft}
\vspace{-10pt}if N is not prime a$^{N-1}$ = 1 mod N $\leq$ half values of a $\textless$ N

\subsubsection{Lagrange's Prime Theorem}
Let $\pi$(x) be the \# of primes $leq$ x, then\\
$\pi$(x) $\approx$ $\frac{x}{ln(x)}$, or more precisely lim$_{x \rightarrow \infty} \frac{\pi(x)}{(\frac{x}{ln(x)})} = 1$\\

\subsubsection{Modular Exponentiation}
x$^{y}$ mod N $\rightarrow$ start with repeated squaring mod N\\
x mod N $\rightarrow$ x$^{2}$ mod N $\rightarrow$ (x$^{2}$)$^{^{2}}$ $\cdots$ x$^{log_{2}y}$ mod N\\
each step takes O(log$^{2}N)$ times to compute and \\there are log$_{2}$y steps, $\therefore$ $\in$ O(n$^{3}$),\\ where n is the \# of bits in N

\subsubsection{Formal Limit Proof}
\begin{flushleft}\vspace{-10pt}\[
\hspace{-50pt}lim_{n \rightarrow \infty}\frac{f(n)}{g(n)}
\begin{cases}
\hspace{3pt} \geq \ 0 \ {\it (\infty)} \Rightarrow \ f(n) \ \in \ \Omega(g(n))\\
\hspace{3pt} \textless \ \infty \ {\it (0)} \Rightarrow \ f(n) \ \in \ O(g(n))\\
\hspace{3pt} = c_{_{\mid 0 \textless c \textless \infty}} \Rightarrow \ f(n) \ \in \ \Theta(g(n))\\
\end{cases}
\]
\end{flushleft}

\subsubsection{Logarithm Tricks}
log$_{b}x^{p} = plog_{b}x$\\
$\frac{ln(x)}{ln(m)}$ = log$_{m}$x\\
x$^{log_{b}y} = y^{log_{b}x}$\\

\subsubsection{Complexity}
\hspace{3pt} $\cdot$ {\it f} $\in$ O({\it g}) if {\it f} $\leq$ c$\cdot${\it g}\\
\hspace{3pt} $\cdot$ {\it f} $\in$ $\Omega$({\it g}) if {\it f} $\geq$ c$\cdot${\it g}\\
\hspace{3pt} $\cdot$ {\it f} $\in$ $\Theta$({\it g}) if {\it f} $\in$ O({\it g}) \& $\Omega$({\it g})\\\vspace{1pt}
{\scriptsize \textit{{\vspace{1pt}Hierarchy:}}\\
\hspace{3pt} $\cdot$ Exponential\\
\hspace{3pt} $\cdot$ Polynomial\\
\hspace{3pt} $\cdot$ Logarithmic\\
\hspace{3pt} $\cdot$ Constant}\\


\subsubsection{Master's Theorem}
T(n) = aT($\frac{n}{b}$) + O(n$^{d}$), if a $\textgreater0, b\textgreater 1, d \geq$ 0\\
\begin{flushleft}\vspace{-16pt}
\[
\hspace{-100pt}T(n)=\begin{cases}
	O(n^{d})$ if d $\textgreater$ log$_{b}a\\
	O(n^{d}log_{b}n)$ if d = log$_{b}a\\
	O(n^{log_{b}n})$ if d $\textless$ log$_{b}a\\
            \end{cases}
\]
\end{flushleft}

\subsubsection{Volker Strassen}
{\scriptsize {\it faster matrix multiplication...}}\\
\[
X = 
\begin{bmatrix}
    A & B\\
    C & D\\
\end{bmatrix}
\hspace{3pt} \times \hspace{3pt} Y = 
\begin{bmatrix}
    E & F\\
    G & H\\
\end{bmatrix}
= 
\begin{bmatrix}
    AE+BG & AF+BH\\
    CE+DG & CF+DH\\
\end{bmatrix}
\]
$\in$ O(n$^{3}$) with recurrence T(n)=8T($\frac{n}{2}$)+O(n$^{2}$)\\
but thanks to Stassen... \\
\[XY = 
\begin{bmatrix}
    P_{5}+P_{4}-P_{2}+P_{6} & P_{1}+P_{2}\\
    P_{3}+P_{4} & P_{1}+P_{5}-P_{3}+P_{7}\\
\end{bmatrix}
\]
{\scriptsize P$_{1}$ = A(F-H) \hspace{3pt} P$_{2}$ = (A+B)H \hspace{3pt} P$_{3}$ = (C+D)E \hspace{3pt} P$_{4}$ = D(G-E) \\
P$_{5}$ = (A+D)(E+H) \hspace{3pt} P$_{6}$ = (B-D)(G+H) \hspace{3pt} P$_{7}$ = (A-C)(E+F)}\\\vspace{3pt}
$\in$ O(n$^{log_{2}7}$) $\approx$ O(n$^{2.81}$) with recurrence T(n)=7T($\frac{n}{2}$)+O(n$^{2}$)\\

\subsubsection{Polynomial Multiplication}
{\it A(x) = a$_{0}+a_{1}x+\cdots+a_{d}x^{d}$ \hspace{3pt} B(x) = b$_{0}+b_{1}x+\cdots+b_{d}x^{d}$\\
C(x)=A(x)$\times$B(x) = a$_{0}b_{k}+a_{1}b_{k-1}+\cdots+a_{k}b_{0} = \sum_{i=0}^{k} a_{i}b_{k-i}$} 

\subsubsection{Fast Fourier Transform}
complex n$^{th}$ roots of unity are given by $\omega = e^{\frac{2\pi i }{n}}$, $\omega^{2}, \omega^{3}, \cdots$\\
\hspace{10pt} $\textless$ values $\textgreater$ = FFT($\textless$ coefficients $\textgreater$, $\omega$)\\
\hspace{10pt} $\textless$ coefficients $\textgreater$ = $\frac{1}{n}$ FFT($\textless$ values $\textgreater$, $\omega^{-1}$) $\in$ O(nlogn)\\
\vspace{2pt}Vandermonde Matrix, %M$_{n}(\omega)$= \hspace{3pt} 
%\[{\scriptsize
%\begin{bmatrix}
%    1 & 1 & 1 & \cdots & 1 \\
%    1 & \omega & \omega^{2} & \cdots & \omega^{n-1}\\
%    1 & \omega^{2} & \omega^{4} & \cdots & \omega^{2(n-1)}\\
%    1 & \cdots & \cdots & \ddots & \omega^{j(n-1)}\\
%    1 & \omega^{n-1} & \omega^{2(n-1)} & \cdots & \omega^{(n-1)(n-1)}\\
%\end{bmatrix}
%{\it where (j,k)^{th} entry \ is \ \omega^{jk}}
%}\]
\[{\scriptsize M_{n}(\omega) =
    \begin{bmatrix}
        1 & 1 & 1 & \cdots & 1\\
        1 & \omega & \omega^{2} & \cdots & \omega^{n-1}\\
        1 & \omega^{2} & \omega^{4} & \cdots & \omega^{2(n-1)}\\
        \vdots & \vdots & \vdots & \ddots & \vdots\\
        1 & \omega^{j} & \omega^{2j} & \cdots & \omega^{(n-1)j}\\
        \vdots & \vdots & \vdots & \ddots & \vdots\\
        1 & \omega^{(n-1)} & \omega^{2(n-1)} & \cdots & \omega^{(n-1)(n-1)}
\end{bmatrix}}\]

\subsubsection{Graphs}
{\scriptsize\hspace{5pt} $\cdot$ \textit{\textbf{graph}} -- set of nodes \& edges between select nodes\\
\hspace{5pt} $\cdot$ \textit{\textbf{tree}} -- a connected graph with no cycles\\
\hspace{5pt} $\cdot$ \textit{\textbf{tree edge}} -- part of DFS forest\\
\hspace{5pt} $\cdot$ \textit{\textbf{forward edge}} -- edge leading from node $\rightarrow$ non-child descendant\\
\hspace{5pt} $\cdot$ \textit{\textbf{back edge}} -- edge leading back to previously visited node\\
\hspace{5pt} $\cdot$ \textit{\textbf{cross edge}} -- edge leading to neither descendant nor ancestor\\
{\it Given an Edge (u,v):}\\
\hspace{5pt} $\cdot$ \textit{\textbf{tree/forward edge}}: {\it pre(u) $\textless$ pre(v) $\textless$ post(v) $\textless$ post(u)} \\
\hspace{5pt} $\cdot$ \textit{\textbf{back edge}}: {\it pre(v) $\textless$ pre(u) $\textless$ post(u) $\textless$ post(v)} \\
\hspace{5pt} $\cdot$ \textit{\textbf{cross edge}}: {\it pre(v) $\textless$ post(v) $\textless$ pre(u) $\textless$ post(u)}\\
{\it Properties:}\\
\hspace{5pt} $\cdot$ a tree on n nodes has n-1 edges\\
\hspace{5pt} $\cdot$ any connected undirected graph with $\mid$E$\mid$ = $\mid$V$\mid$-1 edges is a tree\\
\hspace{5pt} $\cdot$ a directed graph has a cycle iff its DFS reveals a back edge\\
\hspace{5pt} $\cdot$ every DAG has at least 1 source \& 1 sink\\
\hspace{5pt} $\cdot$ in a DAG, every edge leads to a vertex with lower post \#\\
\hspace{5pt} $\cdot$ every directed graph is a DAG of its SCCs\\
\hspace{5pt} $\cdot$ acyclic/linearizability/absence of back edges are all same property\\
\hspace{5pt} $\cdot$ any path of DAG, vertices appear in increasing linearized order\\\hspace{10pt}(linearize, topological sort DAG by DFS, then visit vertices in \\\hspace{10pt}sorted order, updating edges out of each)\\
\hspace{5pt} $\cdot$ if explore starts at u, it will terminate when all nodes \\\hspace{10pt}reachable from u have been visited\\
\hspace{5pt} $\cdot$ node receives highest post order in DFS must lie in source SCC\\
\hspace{5pt} $\cdot$ if C \& C$^{'}$ are SCCs \& $\exists$ an edge from a node in C $\rightarrow$ C$^{'}$ $\Rightarrow$ \\\hspace{12pt}highest post order number in C $\textgreater$ than C$^{'}$'s highest post \#\\
\hspace{5pt} $\cdot$ min edges to make graph strongly connected with n-sinks \& \\\hspace{12pt}m-sources$\rightarrow${\it max(n,m)}\\
{\it Linearize (topologically sort from earliest $\rightarrow$ latest)}\\
\hspace{5pt} $\cdot$ perform tasks in decreasing order of their post numbers (DFS)\\
\hspace{5pt} $\cdot$ or find a source, output it, delete it, repeat until empty\\
{\it Algorithm to Decompose G into SSCs}\\
{\addtolength{\leftskip}{4mm}
\texttt{Run DFS on G$^{R}$, then run DFS on G, every node it reaches is in that SCC, pick next vertex to run DFS from in order of decreasing post \#s discovered from DFS ordering on G$^{R}$}\\}
{\it Shortest/Longest path in a DAG}\\
{\addtolength{\leftskip}{4mm}
\texttt{Linearize DAG by DFS, visit vertices in sorted order, updating edges out of each. Note for longest paths, just negate all edge lengths.}\\}}

\subsubsection{Depth First Search}
{\scriptsize {\it discovers what nodes are reachable from a vertex $\in$ O($\mid$V$\mid$+$\mid$E$\mid$)}}\\
\texttt{\textbf{explore}(G, v):\\
\hspace{3pt} v.visit = true\\
\hspace{3pt} previsit(v)\\
\hspace{3pt} for each edge (v, u) in E:\\
\hspace{9pt} if u.visit = false: explore(G, u)\\
\hspace{3pt} postvisit(v)}\\
 
\texttt{\textbf{dfs}(G):\\
\hspace{3pt} for all v $\in$ V: v.visit = false\\
\hspace{3pt} for all v $\in$ V: if v.visit = false: explore(G,v)}
  
\subsubsection{Breadth First Search}
{\scriptsize {\it  $\in$ O($\mid$V$\mid$+$\mid$E$\mid$)\\}}
\texttt{\textbf{bfs}(G, s):\\
\hspace{3pt} for all u $\in$ V: dist(u) = $\infty$\\
\hspace{12pt} dist(s) = 0\\
\hspace{3pt} Q = [s] (queue containing just {\it s})\\
\hspace{3pt} while Q is not empty:\\
\hspace{12pt} u = eject(Q)\\
\hspace{12pt} for all edges (u,v) $\in$ E:\\
\hspace{20pt} if dist(v) = $\infty$:\\
\hspace{27pt} inject(Q,v)\\
\hspace{27pt} dist(v) = dist(u) + 1}

\subsubsection{Dijkstra's Algorithm}
{\scriptsize {\it shortest path algorithm (+ edge weights) $\in$ O(($\mid$V$\mid$+$\mid$E$\mid$)log$\mid$V$\mid$)\\}}
\texttt{\textbf{dijkstra}(G, l, s):\\
\hspace{3pt} for all u $\in$ V: \\
\hspace{10pt} dist(u) = $\infty$\\
\hspace{10pt} prev(u) = nil\\
\hspace{3pt} dist(s) = 0\\
\hspace{3pt} H = makequeue(V) (using dist-values as keys)\\
\hspace{3pt} while H is not empty:\\
\hspace{10pt} u = deletemin(H)\\
\hspace{10pt} for all edges (u,v) $\in$ E:\\
\hspace{15pt} if dist(v) $\textgreater$ dist(u) + l(u,v):\\
\hspace{23pt} dist(v) = dist(u) + l(u,v)\\
\hspace{23pt} prev(v) = u\\
\hspace{23pt} decreasekey(H,v)}
\includegraphics[scale=.2]{heaps.png}

\subsubsection{Bellman-Ford Algorithm}
{\scriptsize {\it shortest path algorithm (+/- edge weights) $\in$ O(($\mid$V$\mid$$\cdot$$\mid$E$\mid$)\\}}
\texttt{\textbf{bellman\_ford}(G, l, s):\\
\hspace{3pt} for all u $\in$ V: \\
\hspace{10pt} dist(u) = $\infty$\\
\hspace{10pt} prev(u) = nil\\
\hspace{3pt} dist(s) = 0\\
\hspace{3pt} repeat $\mid$V$\mid$ - 1 times:\\
\hspace{10pt} for all e $\in$ E:\\
\hspace{20pt} update(e)\\
\textbf{update}(u,v):\\
\hspace{3pt} min\{dist(v), dist(u) + l(u,v)\}\\
{\it note: negative cycle exists if any edge distance value is reduced on $\mid$V$\mid^{th}$ iteration}}

\subsubsection{Kruskal's Algorithm}
{\scriptsize {\it Minimum Spanning Tree Algorithm $\in$ O($\mid$E$\mid$log$\mid$V$\mid$).}\\
{\texttt{Starts with an empty graph \& selects edges from E repeatedly with lightest weight that does not produce a cycle  Uses disjoint sets to determine whether a cycle exists in amortized constant time ({\it see disjoint set data structure}).}}}

\subsubsection{Cut Property}
{\scriptsize \it Suppose edges X are part of a minimum spanning tree G. Pick any subset of nodes S for which X does not cross between S \& V-S, \& let e be the lightest edge across this partition. Then X U e is part of some MST. }

\subsubsection{Disjoint Set Data Structure}
{\scriptsize {\it Directed tree, nodes are elements of the set, each has parent pointer eventually leading to the root of the tree whose parent pointer is itself.}\\
\hspace{5pt} $\cdot$ a root node with rank k is created merging two trees rank k-1\\
\hspace{5pt} $\cdot$ any root node of rank k has $\geq$ 2$^{k}$ nodes in its tree\\
\hspace{5pt} $\cdot$ if there are n elements there are at most n/$2^{k}$ nodes of rank k\\
\hspace{5pt} $\cdot$ trees have height $\leq$ logn - upper-bound on run time of find \& \\\hspace{10pt}union operations\\
\hspace{5pt} $\cdot$ path compression reduces average time per operation to \\\hspace{10pt}amortized O(1)} 

\subsubsection{Prim's Algorithm}
{\scriptsize {\it Minimum Spanning Tree algorithm $\in$ O(($\mid$V$\mid$+$\mid$E$\mid$)log$\mid$V$\mid$)}\\
\texttt{Initialize a tree with a single vertex, chosen arbitrarily from the graph.
Grow the tree by one edge: Of the edges that connect the tree to vertices not yet in the tree, find the minimum-weight edge, and transfer it to the tree.
Repeat until all vertices are in tree.}}

\subsubsection{Huffman Encoding}
{\scriptsize {\it A prefix-free encoding represented by a full binary tree, generated by a path from root to leaf, interpreting left as 0 \& right as 1.}\\
\texttt{cost of tree = $\sum_{i=1}^{n}f_{i}\cdot$(depth of ith symbol in tree)}\\
or \texttt{cost of tree = sum of frequencies of all leaves and internal nodes except root.}\\
Construct tree greedily: \texttt{Start with two symbols with smallest frequencies, continue branching upward constructing tree with next two smallest frequencies (consider sums also) until there are none left.}}

\subsubsection{Horn Formulas}
{\scriptsize \hspace{5pt} $\cdot$ \textit{\textbf{literal}} -- x or, its negation, $\neg$x\\
\hspace{5pt} $\cdot$ \textit{\textbf{clause types}}: \\\hspace{20pt} $\cdot$ \textit{\textbf{Implication:} left side is AND of \# positive literals \& \\\hspace{25pt}right side is a single positive literal {\textbf         (z $\wedge$ w) $\Rightarrow$ u}\\\hspace{20pt} $\cdot$ \textbf{Pure Negative Clause:} consist of OR of \# of negative \\\hspace{25pt}literals} ($\bar{u} \vee \bar{v} \vee \bar{y}$)\\
Horn's Formula: \texttt{Set all variables to false. While there is an implication that isn't satisfied: set right-hand variable of implication to true. If all pure negative clauses are satisfied: return the assignment. else return not satisfiable.}}

\subsubsection{Set Cover}
{\scriptsize {\it Choose a selection of sets whose union is B\\} \texttt{Pick set S$_{i}$ with largest number of uncovered elements. Repeat until all elements in B are covered.\\}{\it Note: if B contains n elements and optimal cover consists of k sets, then greedy algorithm will use at most k$\cdot$ln(n) sets.}}

\subsubsection{Dynamic Programming}
{\scriptsize {\bf Longest Increasing Subsequence:} O(n$^{2}$)\\
The following algorithm starts at one side of the list and finds the max length of sequences terminating at that given node, recursively following backlinks. Then given all the lengths of paths terminating at that given node choose the max length. Without memoization, this solution would be exponential time.\vspace{-10pt}
\begin{verbatim}
L = {}
for j=1,2,...,n:
    L[j] = 1+max{L[i]:(i,j) in E}
    # The (i,j) represents all the edges that go from
    # node i to node j.
return max(L)
\end{verbatim}}

{\scriptsize {\bf Edit Distance (Spelling Suggestions):} O(nm)\\
This algorithm works by choosing the min of the options for every combination of letters/empty spaces.\vspace{-10pt}
\begin{verbatim}
S _ N O W Y
S U N N _ Y
\end{verbatim}\vspace{-20pt}
\begin{verbatim}
for i = 0,1,2,...,m:
    E(i,0) = i
for j = 1,2,...,n:
    E(0,j) = j
for i = 1,2,...,m:
    for j = 1,2,...,n:
        E(i,j) = min{E(i-1,j)+1,E(i,j-1)+1,
        E(i-1,j-1) +diff(i,j)}
return E(m,n)
\end{verbatim}}

{\scriptsize {\bf Knapsack:} O(nW)\\
Items have a weight and a value, goal being to maximize the value within a given weight. (The amount you can carry in your knapsack)\\
{\it with repetition:}\\
\hspace{15pt}K($\omega$)=max$_{\texttt{items}}$\{K($\omega-\omega_{\texttt{item}}$) + v$_{i}$\} $\mid$ {\tiny for 1 $\rightarrow$ W}\\
{\it without repetition:}\\
\hspace{15pt}K($\omega,j)=max_{\texttt{items}}\{K(\omega-\omega_{j},j-1)+v_{j},K(\omega, j-1)$\}$\mid${\tiny for 1 $\rightarrow$ W,j}}

{\scriptsize {\bf Chain Matrix Multiplication:} O(n$^{3}$)\\
C(i,j)=min\{C(i,k)+C(k+1,j)+m$_{i-1}$$\cdot$ m$_{k}$ $\cdot$ m$_{j}$\}}

{\scriptsize {\bf Floyd-Warshall Algorithm for Shortest Paths:} O($\mid$V$\mid$$^{3}$)\\\vspace{-10pt}
\begin{verbatim}
for i=1 to n:
    for j=1 to n:
        dist(i,j,0) = infinity
        for all (i,j) in E:
            dist(i,j,0) = l(i,j)
for k = 1 to n:
    for i = 1 to n:
        for j = 1 to n:
            dist(i,j,k) = min{dist(i,k,k-1)+
                        dist(k,j,k-1),dist(i,j,k-1)}
\end{verbatim}}

{\scriptsize {\bf Traveling Salesman Problem:} O(n$^{2}2^{n}$)\\\vspace{-10pt}
\begin{verbatim}
C({1},1)=0
for s = 2 to n:
    for all subsets S in {1,2,...,n} of size s and has l:
        C(S,1) = infinity
        for all j in S,j != 1:
            C(S,j) = min{C(S-{j},i)+dij:i in S,i not in j}
return min over j, C({1,...,n},j)+dj1
\end{verbatim}}

\subsubsection{Linear Programming}
{\scriptsize {\it Given a set of variables, we want to assign real values to satisfy set of linear equations/inequalities and maximize/minimize given linear objective function. General rule of linear programs that the optimum is achieved at a vertex of feasible region. Exceptions are if there is no optimum because of 1. infeasibility (constraints so tight it is impossible to satisfy all) or 2. unbounded (constraints are so loose it is possible to achieve arbitrarily high objective value)}

\subsubsection*{Properties of Linear Programs}
\begin{enumerate}
    \item To turn a maximization problem into a minimization (or vice versa) multiply the coefficients of the objective function by -1.
    \item To turn an inequality constraint like \(\sum_{i=1}^{n}a_i x_i \leq b\) into an equation, introduce a new variable s and use, \(\sum^n_{i=1}a_{i}x_{i}+s=b\), \(s \geq 0\) (s is known as a slack variable)
    \item To change an equality constraint into inequalities rewrite \(ax=b\),\\
        as \(ax \leq b \texttt{ and } ax \geq b\)
    \item Dealing with variable x that is unrestricted in sign, introduce 2 non-negative variables x$^{+}$, x$^{-}$ $\geq$ 0 \& replace x in the constraints by x$^{+} - x^{-}$
    \item If a linear program has an unbounded value then its dual must be infeasible.
\end{enumerate}

\subsubsection*{Solving Linear Programs with the Simplex method}
typically polynomial time, but in worst case, exponential
\begin{verbatim}
let v be any vertex of the feasible region
while there is a neighbor v' of v with a better value:
    set v = v'
return v
\end{verbatim}\vspace{-8pt}}

\subsubsection*{Max Flow Algorithm}
{\scriptsize Size of the flow is the total quantity sent from s to t, in other words the quantity leaving s. Size of max flow = $\sum$ flows from s to t. $\in$ O($\mid$V$\mid$$\mid$E$\mid^{2}$)
Start with zero flow.\\
Repeat:\\
Choose an appropriate path from s to t, and increase flow along the edges of this path as much as possible.

\subsubsection*{Max Flow Min Cut Theorem}
The size of the maximum flow in a network equals the capacity of the smallest (s,t)-cut, where and (s,t)-cut partitions the vertices into two disjoint groups L and R such that s (start) is in L and t (goal) is in R.}



\subsubsection*{Proving Optimality of a Linear Program Result, Duality}
max \(x_{1}+6x_{2}\)\\
\(\begin{array}{lc}
    \texttt{Inequality} & \texttt{multiplier}\\
    x_{1} \leq 200 & y_{1}\\
    x_{2} \leq 300 & y_{2}\\
    x_{1}+x_{2} \leq 400 & y_{3}\\
    x_{1},x_{2} \geq 0
\end{array}\)
\((y_{1}+y_{2})x_{1} + (y_{2}+y_{3})x_{2} \leq 200y_{1}+300y_{2}+400y_{3}\)\\
resulting in,\\
\(\texttt{min } 200y_{1}+300y_{2}+400y_{3}\)\\
\(y_{1}+y_{3} \geq 1\)\\
\(y_{2}+y_{3} \geq 6\)\\
\(y_{1},y_{2},y_{3} \geq 0\)\\
Which both result in the same optimum (via simplex) thus proving optimality.

\subsubsection*{Zero Sum Games}
% todo need to add information about zero sum games

\subsubsection*{Bipartite Matching}
example is given a graph with two sets, Girls and Boys where lines between the sets are who likes who. Find a graph where every Boy and Girl is matched up with someone they like. This problem reduces to a maximum-flow problem solvable by linear programming.

\subsection*{NP-Complete Problems}
\begin{center}
\begin{tabular}{|c|c|}
    \hline
    Hard problems(NP-complete) & Easy problems (in P)\\
    \hline
    \hline
    3SAT & 2SAT, HORN SAT\\
    Traveling Salesman Problem & Minimum Spanning Tree\\
    Longest Path & Shortest Path\\
    3D Matching & Bipartite Matching\\
    Knapsack & Unary Knapsack\\
    Independent Set & Independent Set on trees\\
    Integer Linear Programming & Linear Programming\\
    Rudrata Path & Euler Path\\
    Balanced Cut & Minimum Cut\\
    \hline
\end{tabular}
\end{center}
SAT = Search Algorithm Time
\begin{center}
\includegraphics[scale=.4]{red.png}
\end{center}

\includegraphics[scale=.115]{dual.jpeg}
\includegraphics[scale=.115]{zero.jpg}




% You can even have references
\rule{0.3\linewidth}{0.25pt}
\scriptsize
\bibliographystyle{abstract}
\bibliography{refFile}
\end{multicols}
\end{document}
